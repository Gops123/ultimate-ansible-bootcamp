\chapter{Getting Started with Ansible (Ad Hoc Server Management)}

\section{Creating Project Specific Ansible Configuration}

The default configurations for ansible resides at /etc/ansible/ansible.cfg. Instead of relying on defaults, we are going to creates  a custom configuration file for our project. The advantage with that is we could take this configurations on any host and execute it the same way, without touching the default system configurations.  This custom configurations will essentially  override the values in /etc/ansible/ansible/cfg.

\subsection{Ansible configuration file}

Change into /vagrant/code/chapter3 directory on your ansible host. Create a file called ansible.cfg  Add  the following contents to the file.

%= lang:text
\begin{code}
remote_user = vagrant
inventory   = myhosts.ini
\end{code}

\section{Creating Host Inventory}

Create a new file called \emph{myhosts.ini} in the same directory.\newline
Let's create three groups as follows,

%= lang:text
\begin{code}
[local]
localhost ansible_connection=local

[app]
192.168.61.12
192.168.61.13

[db]
192.168.61.11
\end{code}

\begin{itemize}
\item First group contains the localhost, the control host. Since it does not need to be connected over ssh, it mandates we add ansible\_connection=local option
\item Second group contains  Application Servers. We will add  two app servers to this group.
\item Third group holds the information about the database servers.
\end{itemize}

\section{Setting up passwordless ssh access to inventory hosts}

\subsection{Generating ssh keypair on control host}

Now on control host, execute the following command

%= lang:text
\begin{code}
ssh-keygen -t rsa
\end{code}

Now press enter for the passphrase and other queries.

[Output]\newline
%= lang:text
\begin{code}
Generating public/private rsa key pair.
Enter file in which to save the key (/root/.ssh/id_rsa):
Created directory '/root/.ssh'.
Enter passphrase (empty for no passphrase):
Enter same passphrase again:
Your identification has been saved in /root/.ssh/id_rsa.
Your public key has been saved in /root/.ssh/id_rsa.pub.
The key fingerprint is:
c5:a5:6d:60:56:5a:7b:3c:60:23:b5:0f:1b:cf:f9:fd root@ansible
The key's randomart image is:
+--[ RSA 2048]----+
|          =oO    |
|         + X *   |
|          = B +  |
|         . . O o |
|        S   . =  |
|               ..|
|                o|
|                .|
|                E|
+-----------------+
\end{code}

\subsection{Copying public key to inventory hosts}

Copy public key of control node to other hosts  

%= lang:text
\begin{code}
ssh-copy-id vagrant@192.168.61.11

ssh-copy-id vagrant@192.168.61.12

ssh-copy-id vagrant@192.168.61.13

ssh-copy-id vagrant@192.168.61.14
\end{code}

See this example output to verify with your output

%= lang:text
\begin{code}
The authenticity of host '192.168.61.11 (192.168.61.11)' can't be established.
RSA key fingerprint is 32:7f:ad:d7:da:63:32:b6:a9:ff:59:af:09:1e:56:22.
Are you sure you want to continue connecting (yes/no)? yes
Warning: Permanently added '192.168.61.11' (RSA) to the list of known hosts.
\end{code}

The password for user \emph{vagrant} is \emph{vagrant}  

\subsection{Validate the passwordless login}

Let us check the connection of control node with other hosts

%= lang:text
\begin{code}
ssh vagrant@192.168.61.11

ssh vagrant@192.168.61.12

ssh vagrant@192.168.61.13

ssh vagrant@192.168.61.14
\end{code}

\subsection{Ansible ping}

We will use Ansible to make sure all the hosts are reachable

%= lang:text
\begin{code}
ansible all -m ping
\end{code}

[Output]\newline
%= lang:text
\begin{code}
192.168.61.13 | SUCCESS => {
    "changed": false,
    "ping": "pong"
}
192.168.61.11 | SUCCESS => {
    "changed": false,
    "ping": "pong"
}
192.168.61.12 | SUCCESS => {
    "changed": false,
    "ping": "pong"
}
localhost | SUCCESS => {
    "changed": false,
    "ping": "pong"
}
\end{code}

\section{Ad Hoc commands:}

Try running following \emph{fire-and-forget} Ad-Hoc commands\ldots{}

\subsection{Run \emph{hostname} command on all hosts}

Let us print the hostname of all the hosts

%= lang:text
\begin{code}
ansible all -a hostname
\end{code}

[Output]\newline
%= lang:text
\begin{code}
localhost | SUCCESS | rc=0 >>
ansible

192.168.61.11 | SUCCESS | rc=0 >>
db

192.168.61.12 | SUCCESS | rc=0 >>
app

192.168.61.13 | SUCCESS | rc=0 >>
app
\end{code}

\subsection{Check the \emph{uptime}}

How long the hosts are \emph{up}?

%= lang:text
\begin{code}
ansible all -a uptime
\end{code}

[Output]\newline
%= lang:text
\begin{code}
localhost | SUCCESS | rc=0 >>
 13:17:13 up  2:21,  1 user,  load average: 0.16, 0.03, 0.01

192.168.61.12 | SUCCESS | rc=0 >>
 13:17:14 up  1:50,  2 users,  load average: 0.00, 0.00, 0.00

192.168.61.13 | SUCCESS | rc=0 >>
 13:17:14 up  1:47,  2 users,  load average: 0.00, 0.00, 0.00

192.168.61.11 | SUCCESS | rc=0 >>
 13:17:14 up  1:36,  2 users,  load average: 0.00, 0.00, 0.00
\end{code}

\subsection{Check memory info on app servers}

Does my app servers have any disk space \emph{free}?

%= lang:text
\begin{code}
ansible app -a free
\end{code}

[Output]\newline
%= lang:text
\begin{code}
192.168.61.13 | SUCCESS | rc=0 >>
             total       used       free     shared    buffers     cached
Mem:        372916     121480     251436        776      11160      46304
-/+ buffers/cache:      64016     308900
Swap:      4128764          0    4128764

192.168.61.12 | SUCCESS | rc=0 >>
             total       used       free     shared    buffers     cached
Mem:        372916     121984     250932        776      11228      46336
-/+ buffers/cache:      64420     308496
Swap:      4128764          0    4128764
\end{code}

\subsection{Installing packages}

Let us \emph{install} Docker on app servers

%= lang:text
\begin{code}
ansible app -a "yum install -y docker-engine"
\end{code}

This command will fail.

[Output]\newline
%= lang:text
\begin{code}
192.168.61.13 | FAILED | rc=1 >>
Loaded plugins: fastestmirror, prioritiesYou need to be root to perform this command.

192.168.61.12 | FAILED | rc=1 >>
Loaded plugins: fastestmirror, prioritiesYou need to be root to perform this command.
\end{code}

Run the following command with \emph{sudo} permissions.

%= lang:text
\begin{code}
ansible app -s -a "yum install -y docker-engine"
\end{code}

This will install docker in our app servers

[Output]\newline
%= lang:text
\begin{code}
192.168.61.12 | SUCCESS | rc=0 >>
Loaded plugins: fastestmirror, priorities
Setting up Install Process
Loading mirror speeds from cached hostfile
 * base: mirrors.nhanhoa.com
 * epel: mirror.rise.ph
 * extras: mirror.fibergrid.in
 * updates: mirror.fibergrid.in
283 packages excluded due to repository priority protections
Resolving Dependencies
--> Running transaction check
---> Package docker-engine.x86_64 0:1.7.1-1.el6 will be installed
--> Finished Dependency Resolution

Dependencies Resolved

================================================================================
 Package             Arch         Version              Repository          Size
================================================================================
Installing:
 docker-engine       x86_64       1.7.1-1.el6          local_docker       4.5 M
                              [...]
Total download size: 4.5 M
Installed size: 19 M
Downloading Packages:
Running rpm_check_debug
Running Transaction Test
Transaction Test Succeeded
Running Transaction
  Installing : docker-engine-1.7.1-1.el6.x86_64                             1/1
  Verifying  : docker-engine-1.7.1-1.el6.x86_64                             1/1

Installed:
  docker-engine.x86_64 0:1.7.1-1.el6

Complete!
\end{code}

\subsection{Running commands one machine at a time}

Do you want a command to run on \emph{one machine at a time} ?

%= lang:text
\begin{code}
ansible all -f 1 -a "free"
\end{code}

\section{Using \emph{modules} to manage the state of infrastructure}

\subsection{Creating users and groups using \emph{user} and \emph{group}}

To create a group

%= lang:text
\begin{code}
ansible app -s -m group -a "name=admin state=present"
\end{code}

[Output]\newline
%= lang:text
\begin{code}
192.168.61.13 | SUCCESS => {
    "changed": true,
    "gid": 501,
    "name": "admin",
    "state": "present",
    "system": false
}
192.168.61.12 | SUCCESS => {
    "changed": true,
    "gid": 501,
    "name": "admin",
    "state": "present",
    "system": false
}
\end{code}

To create a user\ldots{}

%= lang:text
\begin{code}
ansible app -s -m user -a "name=devops group=admin createhome=yes"
\end{code}

This will create user \emph{devops},

[Output]\newline
%= lang:text
\begin{code}
192.168.61.13 | SUCCESS => {
    "changed": true,
    "comment": "",
    "createhome": true,
    "group": 501,
    "home": "/home/devops",
    "name": "devops",
    "shell": "/bin/bash",
    "state": "present",
    "system": false,
    "uid": 501
}
192.168.61.12 | SUCCESS => {
    "changed": true,
    "comment": "",
    "createhome": true,
    "group": 501,
    "home": "/home/devops",
    "name": "devops",
    "shell": "/bin/bash",
    "state": "present",
    "system": false,
    "uid": 501
}
\end{code}

\subsection{Copy a file using \emph{copy} modules}

We will copy file from control node to app servers.

%= lang:text
\begin{code}
ansible app -m copy -a "src=/vagrant/test.txt dest=/tmp/test.txt"
\end{code}

File will be copied over to our app server machines\ldots{}

[Output]\newline
%= lang:text
\begin{code}
192.168.61.13 | SUCCESS => {
    "changed": true,
    "checksum": "3160f8f941c330444aac253a9e6420cd1a65bfe2",
    "dest": "/tmp/test.txt",
    "gid": 500,
    "group": "vagrant",
    "md5sum": "9052de4cff7e8a18de586f785e711b97",
    "mode": "0664",
    "owner": "vagrant",
    "size": 11,
    "src": "/home/vagrant/.ansible/tmp/ansible-tmp-1472991990.29-63683023616899/source",
    "state": "file",
    "uid": 500
}
192.168.61.12 | SUCCESS => {
    "changed": true,
    "checksum": "3160f8f941c330444aac253a9e6420cd1a65bfe2",
    "dest": "/tmp/test.txt",
    "gid": 500,
    "group": "vagrant",
    "md5sum": "9052de4cff7e8a18de586f785e711b97",
    "mode": "0664",
    "owner": "vagrant",
    "size": 11,
    "src": "/home/vagrant/.ansible/tmp/ansible-tmp-1472991990.26-218089785548663/source",
    "state": "file",
    "uid": 500
}
\end{code}

\section{Exercises :}

\begin{enumerate}
\item Add another group called \emph{lb} in inventory with respective host ip
\item Add a user called \emph{joe} in app servers. Make sure that user has a home directory.
\item Install the package git using the correct Ad-Hoc command.
\end{enumerate}