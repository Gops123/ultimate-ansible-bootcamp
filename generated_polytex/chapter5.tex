\chapter{Working with Roles}

In this tutorial we are going to create simple, static role for apache which will,

\begin{itemize}
\item Install \textbf{httpd} package
\item Configure \textbf{httpd.conf}, manage it as a static file
\item Start httpd service
\item Add a notification and a  handler so that whenever the configuration is updated, service is automatically restarted.
\end{itemize}

\section{Creating Role Scaffolding for Apache}

\begin{itemize}
\item Change working  directory to \textbf{/vagrant/code/chap5}
\end{itemize}

%= lang:text
\begin{code}
cd  /vagrant/code/chap5
\end{code}

\begin{itemize}
\item Create roles directory
\end{itemize}

%= lang:text
\begin{code}
mkdir roles
\end{code}

\begin{itemize}
\item Generate role scaffolding using ansible-galaxy
\end{itemize}

%= lang:text
\begin{code}
ansible-galaxy init --offline --init-path=roles  apache
\end{code}

\begin{itemize}
\item Validate
\end{itemize}

%= lang:text
\begin{code}
tree roles/
\end{code}

[Output]\newline
%= lang:text
\begin{code}
roles/
  └── apache
      ├── defaults
      │   └── main.yml
      ├── files
      ├── handlers
      │   └── main.yml
      ├── meta
      │   └── main.yml
      ├── README.md
      ├── tasks
      │   └── main.yml
      ├── templates
      ├── tests
      │   ├── inventory
      │   └── test.yml
      └── vars
          └── main.yml
\end{code}

\section{Writing Tasks to Install and Start Apache Web Service}

We are going to create three different tasks files, one for each phase of application lifecycle

\begin{itemize}
\item Install
\item Configure
\item Start Service
\end{itemize}

To begin with, in this part, we will install and start apache.

\begin{itemize}
\item To install apache, Create \textbf{roles/apache/tasks/install.yml}
\end{itemize}

%= lang:text
\begin{code}
---
- name: Install Apache...
  yum: name=httpd state=latest
\end{code}

\begin{itemize}
\item To start the service, create  \textbf{roles/apache/tasks/start.yml} with the following content
\end{itemize}

%= lang:text
\begin{code}
---
- name: Starting Apache...
  service: name=httpd state=started
\end{code}

To have these tasks being called, include them into main task.

\begin{itemize}
\item Edit roles/apache/tasks/main.yml
\end{itemize}

%= lang:text
\begin{code}
---
# tasks file for apache
- include: install.yml
- include: start.yml
\end{code}

\begin{itemize}
\item Create a playbook for app servers at /vagrant/chap5/app.yml with following contents
\end{itemize}

%= lang:text
\begin{code}
---
- hosts: app
  become: true
  roles:
    - apache
\end{code}

\begin{itemize}
\item Apply app.yml with ansible-playbook
\end{itemize}

%= lang:text
\begin{code}
ansible-playbook app.yml
\end{code}

[Output]\newline
%= lang:text
\begin{code}
PLAY [Playbook to configure App Servers] ***************************************

TASK [setup] *******************************************************************
ok: [192.168.61.12]
ok: [192.168.61.13]

TASK [apache : Install Apache...] **********************************************
changed: [192.168.61.13]
changed: [192.168.61.12]

TASK [apache : Starting Apache...] *********************************************
changed: [192.168.61.13]
changed: [192.168.61.12]

PLAY RECAP *********************************************************************
192.168.61.12              : ok=3    changed=2    unreachable=0    failed=0
192.168.61.13              : ok=3    changed=2    unreachable=0    failed=0
\end{code}

\section{Managing Configuration files for Apache}

\begin{itemize}
\item Copy \textbf{index.html} and \textbf{httpd.conf} from \textbf{chap5/helper} to \textbf{/roles/apache/files/} directory
\end{itemize}

%= lang:text
\begin{code}
cp helper/index.html helper/httpd.conf roles/apache/files/  
\end{code}

\begin{itemize}
\item Create a task file at \textbf{roles/apache/tasks/config.yml} to manage files.
\end{itemize}

%= lang:text
\begin{code}
---
- name: Copying configuration files...
  copy: src=httpd.conf
        dest=/etc/httpd.conf
        owner=root group=root mode=0644

- name: Copying index.html file...
  copy: src=index.html
        dest=/var/www/html/index.html
        mode=0777
\end{code}

\section{Adding Notifications and Handlers}

\begin{itemize}
\item Previously we have create a task in roles/apache/tasks/config.yml to copy over httpd.conf to the app server. Update this file to send a notification to restart  service on configuration update.  You simply have to add the line which starts with \textbf{notify}
\end{itemize}

%= lang:text
\begin{code}
- name: Copying configuration files...
  copy: src=httpd.conf
        dest=/etc/httpd.conf
        owner=root group=root mode=0644
  notify: Restart apache service
\end{code}

\begin{itemize}
\item Create the notification handler by updating   \textbf{roles/apache/handlers/main.yml}
\end{itemize}

%= lang:text
\begin{code}
---
- name: Restart apache service
  service: name=httpd state=restarted
\end{code}

Then run the app role

%= lang:text
\begin{code}
ansible-playbook app.yml
\end{code}

[Output]\newline
%= lang:text
\begin{code}
PLAY [Playbook to configure App Servers] ***************************************

TASK [setup] *******************************************************************
ok: [192.168.61.13]
ok: [192.168.61.12]

TASK [apache : Installing Apache...] *******************************************
ok: [192.168.61.12]
ok: [192.168.61.13]

TASK [apache : Starting Apache...] *********************************************
ok: [192.168.61.13]
ok: [192.168.61.12]

TASK [apache : Copying configuration files...] *********************************
changed: [192.168.61.12]
changed: [192.168.61.13]

TASK [apache : Copying index.html file...] *************************************
changed: [192.168.61.12]
changed: [192.168.61.13]

RUNNING HANDLER [apache : Restart apache service] ******************************
changed: [192.168.61.12]
changed: [192.168.61.13]

PLAY RECAP *********************************************************************
192.168.61.12              : ok=6    changed=3    unreachable=0    failed=0
192.168.61.13              : ok=6    changed=3    unreachable=0    failed=0
\end{code}

\section{Troubleshooting Exercise}

Did the above command added the configuration files and restarted the service? But we have already written \textbf{config.yml}. Troubleshoot why its not being run and fix it before you proceed.

\section{Base Role and Role Nesting}

\begin{itemize}
\item Create a base role with ansible-galaxy utility,
\end{itemize}

%= lang:text
\begin{code}
ansible-galaxy init --offline --init-path=roles base
\end{code}

\begin{itemize}
\item Create tasks for base role by editing  \textbf{/roles/base/tasks/main.yml}
\end{itemize}

%= lang:text
\begin{code}
---
# tasks file for base
# file: roles/base/tasks/main.yml
  - name: create admin user
    user: name=admin state=present uid=5001

  - name: remove dojo
    user: name=dojo  state=present

  - name: install tree
    yum:  name=tree  state=present

  - name: install ntp
    yum:  name=ntp   state=present

  - name: start ntp service
    service: name=ntpd state=started enabled=yes  
\end{code}

\begin{itemize}
\item Define base role as a dependency for  apache role,
\item Update meta data for Apache by editing \textbf{roles/apache/meta/main.yml} and adding the following
\end{itemize}

%= lang:text
\begin{code}
---
dependencies:
 - {role: base}
\end{code}

\section{Creating a Site Wide Playbook}

We will create a site wide playbook, which will call all the plays required to configure the complete infrastructure. Currently we have a single  playbook for App Servers. However, in future we would create many.

\begin{itemize}
\item Create \textbf{site.yml} in /vagrant/chap5 directory and add the following content
\end{itemize}

%= lang:text
\begin{code}
---
# This is a site-wide playbook
# filename: site.yml
- include: app.yml
\end{code}

\begin{itemize}
\item Execute site-wide playbook as
\end{itemize}

%= lang:text
\begin{code}
ansible-playbook site.yml
\end{code}

[Output]\newline
%= lang:text
\begin{code}
PLAY [Playbook to configure App Servers] ***************************************

TASK [setup] *******************************************************************
ok: [192.168.61.12]
ok: [192.168.61.13]

TASK [base : create admin user] ************************************************
ok: [192.168.61.12]
ok: [192.168.61.13]

TASK [base : remove dojo] ******************************************************
ok: [192.168.61.12]
ok: [192.168.61.13]

TASK [base : install tree] *****************************************************
ok: [192.168.61.13]
ok: [192.168.61.12]

TASK [base : install ntp] ******************************************************
ok: [192.168.61.13]
ok: [192.168.61.12]

TASK [base : start ntp service] ************************************************
ok: [192.168.61.13]
ok: [192.168.61.12]

TASK [apache : Installing Apache...] *******************************************
ok: [192.168.61.13]
ok: [192.168.61.12]

TASK [apache : Starting Apache...] *********************************************
ok: [192.168.61.13]
ok: [192.168.61.12]

TASK [apache : Copying configuration files...] *********************************
ok: [192.168.61.12]
ok: [192.168.61.13]

TASK [apache : Copying index.html file...] *************************************
ok: [192.168.61.12]
ok: [192.168.61.13]

PLAY RECAP *********************************************************************
192.168.61.12              : ok=10   changed=0    unreachable=0    failed=0
192.168.61.13              : ok=10   changed=0    unreachable=0    failed=0
\end{code}

\section{Exercises}

\begin{itemize}
\item Update httpd.conf and change some configuration parameters. Validate the service restarts on configuration updates by applying the sitewide playbook.
\item Pre Task to be run before creating MySQL Role

\begin{itemize}
\item From Ansible Control node run the following command to enable repositories. This is needed in order to install some of the db packages.
\end{itemize}
\end{itemize}

%= lang:text
\begin{code}
ansible db -s -a "cp -r /etc/yum.repos.d/repo.bkp/* /etc/yum.repos.d/"
\end{code}

\begin{itemize}
\item Create a Role to install and configure MySQL server


\begin{itemize}
\item Create role scaffold for mysql  using ansible-galaxy init
\item Create task to install ``mysql-server'' and ``MySQL-python'' packages using yum module
\item Create a task to start mysqld service
\item Manage my.cnf by creating a centralized copy in role and writing a task to copy it to all db hosts. Use helper/my.cnf as a reference. The destination for this file is /etv/my.cnf on db servers.
\item Write a handler to restart the service on configuration change. Add a notification from the copy resource created earlier.
\item Add a dependency on base role in the metadata for mysql role.
\item Create  \textbf{db.yml} playbook for configuring all database servers. Create definitions to configure \textbf{db} group and to apply \textbf{mysql} role.
\end{itemize}
\end{itemize}