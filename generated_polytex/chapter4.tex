\chapter{Learning to Write Playbooks}

In this tutorial we are going to create a simple playbook to add system users, install and start ntp service and some basic utilities.

\section{Creating our first playbook}

\begin{itemize}
\item Change working directory to /vagrant/code/chap4
\end{itemize}

%= lang:text
\begin{code}
cd /vagrant/code/chap4
\end{code}

\begin{itemize}
\item Create playbook.yml and add the content below
\end{itemize}

%= lang:text
\begin{code}
---
  - name: Base Configurations for ALL hosts
    hosts: all
    become: true
    tasks:
      - name: create admin user
        user: name=admin state=present uid=5001

      - name: remove dojo
        user: name=dojo  state=absent

      - name: install tree
        yum:  name=tree  state=present

      - name: install ntp
        yum:  name=ntp   state=present

      - name: start ntp service
        service: name=ntpd state=started enabled=yes
\end{code}

\section{Running the  playbook}

To run the playbook, we are going to execute \textbf{ansible-playbook} command. Lets first examine the options that this command supports.

%= lang:text
\begin{code}
ansible-playbook --help
\end{code}

[Output]\newline
%= lang:text
\begin{code}
Usage: ansible-playbook playbook.yml

Options:
  --ask-become-pass     ask for privilege escalation password
  -k, --ask-pass        ask for connection password
  --ask-su-pass         ask for su password (deprecated, use become)
  -K, --ask-sudo-pass   ask for sudo password (deprecated, use become)
  --ask-vault-pass      ask for vault password
  -b, --become          run operations with become (nopasswd implied)
  --become-method=BECOME_METHOD
                        privilege escalation method to use (default=sudo),
                        valid choices: [ sudo | su | pbrun | pfexec | runas |
                        doas ]
\end{code}

To run the playbook, we could call YAML file as an argument. Since we have already defined the inventory and configurations, additional options are not necessary at this time.

%= lang:text
\begin{code}
ansible-playbook playbook.yml
\end{code}

[Output]\newline
%= lang:text
\begin{code}
PLAY [Base Configurations for ALL hosts] ***************************************

TASK [setup] *******************************************************************
ok: [192.168.61.14]
ok: [192.168.61.11]
ok: [localhost]
ok: [192.168.61.12]
ok: [192.168.61.13]

TASK [create admin user] *******************************************************
changed: [192.168.61.13]
changed: [192.168.61.12]
changed: [localhost]
changed: [192.168.61.11]
changed: [192.168.61.14]

TASK [remove dojo] *************************************************************
changed: [192.168.61.14]
changed: [localhost]
changed: [192.168.61.12]
changed: [192.168.61.11]
changed: [192.168.61.13]

TASK [install tree] ************************************************************
ok: [localhost]
ok: [192.168.61.13]
ok: [192.168.61.12]
ok: [192.168.61.14]
ok: [192.168.61.11]

TASK [install ntp] *************************************************************
changed: [192.168.61.12]
changed: [192.168.61.13]
changed: [192.168.61.11]
changed: [localhost]
changed: [192.168.61.14]

TASK [start ntp service] *******************************************************
changed: [192.168.61.11]
changed: [localhost]
changed: [192.168.61.13]
changed: [192.168.61.12]
changed: [192.168.61.14]
\end{code}

\section{Adding second play in the playbook}

Lets add a second play specific to app servers. Add the following block of code in playbook.yml file and save

%= lang:text
\begin{code}
- name: App Server Configurations
  hosts: app
  become: true
  tasks:
    - name: create app user
      user: name=app state=present uid=5003

    - name: install git
      yum:  name=git  state=present
\end{code}

Run the playbook again\ldots{}

%= lang:text
\begin{code}
ansible-playbook playbook.yml
\end{code}

[Output]\newline
%= lang:text
\begin{code}
PLAY [Base Configurations for ALL hosts] ***************************************

TASK [setup] *******************************************************************
ok: [localhost]
ok: [192.168.61.14]
ok: [192.168.61.13]
ok: [192.168.61.12]
ok: [192.168.61.11]

TASK [create admin user] *******************************************************
ok: [192.168.61.14]
ok: [localhost]
ok: [192.168.61.13]
ok: [192.168.61.12]
ok: [192.168.61.11]

TASK [remove dojo] *************************************************************
ok: [192.168.61.14]
ok: [192.168.61.12]
ok: [192.168.61.11]
ok: [localhost]
ok: [192.168.61.13]

TASK [install tree] ************************************************************
ok: [192.168.61.11]
ok: [192.168.61.12]
ok: [localhost]
ok: [192.168.61.13]
ok: [192.168.61.14]

TASK [install ntp] *************************************************************
ok: [192.168.61.12]
ok: [192.168.61.13]
ok: [192.168.61.14]
ok: [192.168.61.11]
ok: [localhost]

TASK [start ntp service] *******************************************************
ok: [192.168.61.13]
ok: [192.168.61.14]
ok: [localhost]
ok: [192.168.61.11]
ok: [192.168.61.12]

PLAY [App Server Configurations] ***********************************************

TASK [setup] *******************************************************************
ok: [192.168.61.13]
ok: [192.168.61.12]

TASK [create app user] *********************************************************
changed: [192.168.61.12]
changed: [192.168.61.13]

TASK [install git] *************************************************************
ok: [192.168.61.13]
ok: [192.168.61.12]

PLAY RECAP *********************************************************************
192.168.61.11              : ok=6    changed=0    unreachable=0    failed=0
192.168.61.12              : ok=9    changed=1    unreachable=0    failed=0
192.168.61.13              : ok=9    changed=1    unreachable=0    failed=0
192.168.61.14              : ok=6    changed=0    unreachable=0    failed=0
localhost                  : ok=6    changed=0    unreachable=0    failed=0
\end{code}

\section{Limiting the execution to a particular group}

Now run the following command to restrict the playbook execution to \emph{app servers}

%= lang:text
\begin{code}
ansible-playbook playbook.yml --limit app
\end{code}

This will give us the following output, plays will be executed only on app servers\ldots{}

[Output]\newline
%= lang:text
\begin{code}
PLAY [Base Configurations for ALL hosts] ***************************************

TASK [setup] *******************************************************************
ok: [192.168.61.13]
ok: [192.168.61.12]

TASK [create admin user] *******************************************************
ok: [192.168.61.13]
ok: [192.168.61.12]

TASK [remove dojo] *************************************************************
ok: [192.168.61.12]
ok: [192.168.61.13]

TASK [install tree] ************************************************************
ok: [192.168.61.13]
ok: [192.168.61.12]

TASK [install ntp] *************************************************************
ok: [192.168.61.12]
ok: [192.168.61.13]

TASK [start ntp service] *******************************************************
ok: [192.168.61.12]
ok: [192.168.61.13]

PLAY [App Server Configurations] ***********************************************

TASK [setup] *******************************************************************
ok: [192.168.61.13]
ok: [192.168.61.12]

TASK [create app user] *********************************************************
ok: [192.168.61.13]
ok: [192.168.61.12]

TASK [install git] *************************************************************
ok: [192.168.61.12]
ok: [192.168.61.13]

PLAY RECAP *********************************************************************
192.168.61.12              : ok=9    changed=0    unreachable=0    failed=0
192.168.61.13              : ok=9    changed=0    unreachable=0    failed=0
\end{code}

\section{Exercise:}

Create a Playbook with the following specifications,

\begin{itemize}
\item It should apply only on local host (ansible host)
\item Should use become method
\item Should create a \textbf{user} called webadmin with shell as ``/bin/sh''
\item Should install and start \textbf{nginx} service
\item Should \textbf{deploy} a sample html app into the default web root directory of nginx using ansible's \textbf{git} module.


\begin{itemize}
\item Source repo:
https://github.com/schoolofdevops/html-sample-app


\begin{itemize}
\item Deploy Path : /usr/share/nginx/html/app
\item The user to deploy the app would be nginx
\end{itemize}
\end{itemize}
\end{itemize}

\paragraph{\textbf{TODO for Course Creator:}}

\begin{itemize}
\item Fail the task (w.g. service name = ntp).  
\item It will create a retry file\newline
   Use that to feed into to ansible-playbook with --limit option\newline
   e.g.\newline
   \kode{
    ansible-playbook playbook.yml --limit @/tmp/playbook.rerty
  }
\end{itemize}