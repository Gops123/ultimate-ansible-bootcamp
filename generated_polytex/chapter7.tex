\chapter{Control Structures}

In Chapter 7, we will learn about the aspects of conditionals and iterations that affects program's execution flow in Ansible\newline
Control structures are of two different type

\begin{itemize}
\item Conditional
\item Iterative
\end{itemize}

\section{Conditionals}

Conditionals structures allow Ansible to choose an alternate path. Ansible does this by using \emph{when} statements

\section{\textbf{When} statements}

When statement becomes helpful, when you will want to skip a particular step on a particular host

\subsection{Selectively calling install tasks based on platform}

\begin{itemize}
\item Edit \emph{roles/apache/tasks/main.yml},
\end{itemize}

%= lang:text
\begin{code}
---
- include: install.yml
  when: ansible_os_family == 'RedHat'
- include: start.yml
- include: config.yml
\end{code}

\begin{itemize}
\item This will include \emph{install.yml} only if the OS family is Redhat, otherwise it will skip the installation playbook
\end{itemize}

\subsection{Configuring MySQL server based on boolean flag}

\begin{itemize}
\item Edit \emph{roles/mysql/tasks/main.yml} and add when statements,
\end{itemize}

%= lang:text
\begin{code}
---
# tasks file for mysql
- include: install.yml

- include: start.yml
  when: mysql.server

- include: config.yml
  when: mysql.server
\end{code}

\begin{itemize}
\item Edit \emph{db.yml} as follows,
\end{itemize}

%= lang:text
\begin{code}
---
  - name: Playbook to configure DB Servers
    hosts: db
    become: true
    roles:
    - mysql
    vars:
      mysql:
        server: true
        config:
          bind: "{{ ansible_eth1.ipv4.address }}"
\end{code}

\subsection{Adding conditionals in Jinja2 templates}

\begin{itemize}
\item Put the following content in \emph{roles/mysql/templates/my.cnf.j2}
\end{itemize}

%= lang:text
\begin{code}
[mysqld]


datadir={{ mysql['config']['datadir'] }}



socket={{ mysql['config']['socket'] }}


symbolic-links=0
log-error=/var/log/mysqld.log


pid-file={{ mysql['config']['pid']}}


[client]
user=root
password={{ mysql_root_db_pass }}
\end{code}

\begin{itemize}
\item These conditions will run flawlessly, because we have already defined these Variables
\end{itemize}

\subsection{Running One Time Tasks}

\begin{itemize}
\item To see how this works, lets take a look at the code in \emph{roles/mysql/tasks/config.yml}
\end{itemize}

%= lang:text
\begin{code}
        [...]
- name: reset default root password
shell: mysql --user=root --password="{{ MYSQL_DEFAULT_PASS }}" --connect-expired-password mysql < /root/.mysql_reset_pass.sql
run_once: true
ignore_errors: yes
       [...]
\end{code}

\begin{itemize}
\item In some cases there may be a need to only run a task one time and only on one host. This can be achieved by configuring “run\_once” on a task
\end{itemize}

\subsection{Conditional Execution of Roles}

\begin{itemize}
\item This will execute app playbook only if the node is running \textbf{RedHat} family  
\item Update app.yml to restrict role to be run only on RedHat platform.
\end{itemize}

%= lang:text
\begin{code}
---
  - name: Playbook to configure App Servers
    hosts: app
    become: true
    vars:
      fav:
        fruit: mango
    roles:
    - { role: apache, when: ansible_os_family == 'RedHat' }
\end{code}

\begin{itemize}
\item Let's run this code
\end{itemize}

%= lang:text
\begin{code}
ansible-playbook site.yml
\end{code}

[Output]\newline
%= lang:text
\begin{code}
TASK [setup] *****************************************************************
ok: [192.168.61.12]
ok: [192.168.61.13]

TASK [base : create admin user] **********************************************
skipping: [192.168.61.12]
skipping: [192.168.61.13]

TASK [base : remove dojo] ****************************************************
skipping: [192.168.61.12]
skipping: [192.168.61.13]

TASK [base : install tree] ***************************************************
skipping: [192.168.61.12]
skipping: [192.168.61.13]

TASK [base : install ntp] ****************************************************
skipping: [192.168.61.12]
skipping: [192.168.61.13]

TASK [base : start ntp service] **********************************************
skipping: [192.168.61.12]
skipping: [192.168.61.13]

TASK [apache : Installing Apache...] *****************************************
skipping: [192.168.61.12]
skipping: [192.168.61.13]

TASK [apache : Starting Apache...] *******************************************
skipping: [192.168.61.12]
skipping: [192.168.61.13]

TASK [apache : Creating configuration from templates...] *********************
skipping: [192.168.61.12]
skipping: [192.168.61.13]

TASK [apache : Copying index.html file...] ***********************************
skipping: [192.168.61.12]
skipping: [192.168.61.13]
\end{code}

\subsection{Exercise}

Try using \textbf{Debian} instead of \textbf{RedHat} . You shall see app role being skipped altogether. Don't forget to put it back after you try this out.

\section{Iterations}

\subsection{Iteration over list}

\begin{itemize}
\item Create a list of packages


\begin{itemize}
\item Let us create the following list of packages in base role.
\item Edit \emph{roles/base/defaults/main.yml} and put
\end{itemize}
\end{itemize}

%= lang:text
\begin{code}
---
# packages list
demolist:
  packages:
    - atk
    - flac
    - eggdbus
    - pixman
    - polkit
\end{code}

\begin{itemize}
\item Also edit \emph{roles/base/tasks/main.yml} to iterate over this list of items and install packages
\end{itemize}

%= lang:text
\begin{code}
- name: install a list of packages
  yum:
    name: "{{ item }}"
  with_items: {{ demolist.packages }}
\end{code}

\begin{itemize}
\item Let's check the output
\end{itemize}

%= lang:text
\begin{code}
TASK [base : install a list of packages] ***************************************
changed: [192.168.61.12] => (item=[u'atk', u'flac', u'eggdbus', u'polkit', u'pixman'])
changed: [192.168.61.13] => (item=[u'atk', u'flac', u'eggdbus', u'polkit', u'pixman'])
\end{code}

\subsection{Iterating over a Hash Table/Dictionary}

\begin{itemize}
\item This iteration can be done with using \textbf{with\_dict} statement, let us see how
\item Edit \emph{group\_vars/all} file from the \textbf{parent directory} and define a dictionary of mysql databases and users to be created
\end{itemize}

%= lang:text
\begin{code}
---
  fav:
    color: blue
    fruit: peach
  mysql_bind: "{{ ansible_eth0.ipv4.address }}"
  mysql:
    databases:
      infinity:
        state: present
      peace:
        state: present
    users:
      dojo:
        pass: PassWord@1234
        host: '%'
        priv: '*.*:ALL'
        state: present
      koko:
        pass: f8Usg3ord@1we28
        host: '%'
        priv: '*.*:ALL'
        state: present
\end{code}

\begin{itemize}
\item \textbf{Append} the following iteration in \emph{roles/mysql/tasks/config.yml}
\end{itemize}

%= lang:text
\begin{code}
- name: create mysql databases
  mysql_db:
    name: "{{ item.key }}"
    state: "{{ item.value.state }}"
  with_dict: "{{ mysql['databases'] }}"

- name: create mysql users
  mysql_user:
    name: "{{ item.key }}"
    host: "{{ item.value.host }}"
    password: "{{ item.value.pass }}"
    priv: "{{ item.value.priv }}"
    state: "{{ item.value.state }}"
  with_dict: "{{ mysql['users'] }}"
\end{code}

\begin{itemize}
\item Execute the \emph{db} playbook to verify the output
\end{itemize}

%= lang:text
\begin{code}
ansible-playbook db.yml
\end{code}

\section{Exercises}

\begin{itemize}
\item Define dictionary of properties for a new database user  in group\_vars/all. Observe if it gets created automatically  output by running db.yml playbook. Validate if the user is been actually present by logging on to the mysql server and checking status.
\item Update index.html.j2 to iterate over the dictionary of favorites and generate html content to display it instead of adding multiple lines.
\item Define a hash/dictionary  of apache virtual hosts to be created  and create a template which would iterate over that dictionary and create vhost configurations.
\item Learn about what else you could loop over, as well as how to do so by reading this document http://docs.ansible.com/ansible/playbooks\_loops.html\#id12
\end{itemize}