\chapter{Control Structures}

In Chapter 7, we will learn about the aspects of conditionals and iterations that affects program's execution flow in Ansible\newline
Control structures are of two different type\newline
* Conditional
* Iterative  

\section{Conditionals}

Conditionals structures allow Ansible to choose an alternate path. Ansible does this by using \emph{when} statements\newline
\#\#\# 7.1 \textbf{When} statements\newline
When statement becomes helpful, when you will want to skip a particular step on a particular host

\subsubsection{Selectively calling install tasks based on platform}

\begin{itemize}
\item Edit \emph{roles/apache/tasks/main.yml},
  ```
  ---
\item include: install.yml
when: ansible\_os\_family == `RedHat'
\item include: start.yml
\item include: config.yml
\end{itemize}

```
  * This will include \emph{install.yml} only if the OS family is Redhat, otherwise it will skip the installation playbook  

\subsubsection{Configuring MySQL server based on boolean flag}

\begin{itemize}
\item Edit \emph{roles/mysql/tasks/main.yml} and add when statements,\newline
  ```
  ---
  \# tasks file for mysql
\item include: install.yml
\item include: start.yml
when: mysql.server
\item include: config.yml
when: mysql.server
  ```  
\item Edit \emph{db.yml} as follows,
  ```
  ---


\begin{itemize}
\item name: Playbook to configure DB Servers
hosts: db
become: true
roles:


\begin{itemize}
\item mysql
vars:
mysql:
  server: true
  config:
    bind: ``\{\{ ansible\_eth1.ipv4.address \}\}''
\end{itemize}
\end{itemize}
\end{itemize}

```  

\subsubsection{Adding conditionals in Jinja2 templates}

\begin{itemize}
\item Put the following content in \emph{roles/mysql/templates/my.cnf.j2}\newline
  ```
  [mysqld]
\end{itemize}

\{\% if mysql.config.datadir is defined \%\}
  datadir=\{\{ mysql[`config'][`datadir'] \}\}
  \{\% endif \%\}

\{\% if mysql.config.socket is defined \%\}
  socket=\{\{ mysql[`config'][`socket'] \}\}
  \{\% endif \%\}

symbolic-links=0
  log-error=/var/log/mysqld.log

\{\% if mysql.config.pid is defined \%\}
  pid-file=\{\{ mysql[`config'][`pid']\}\}
  \{\% endif \%\}

[client]
  user=root
  password=\{\{ mysql\_root\_db\_pass \}\}
  ```  

\begin{itemize}
\item These conditions will run flawlessly, because we have already defined these Variables
\end{itemize}

\subsubsection{Running One Time Tasks}

\begin{itemize}
\item To see how this works, lets take a look at the code in \emph{roles/mysql/tasks/config.yml}\newline
  ```
      [\ldots{}]
\item name: reset default root password
  shell: mysql --user=root --password=''\{\{ MYSQL\_DEFAULT\_PASS \}\}'' --connect-expired-password mysql \textless{} /root/.mysql\_reset\_pass.sql
  run\_once: true
  ignore\_errors: yes
     [\ldots{}]
\end{itemize}

``` \newline
  * In some cases there may be a need to only run a task one time and only on one host. This can be achieved by configuring “run\_once” on a task  

\subsubsection{Conditional Execution of Roles}

\begin{itemize}
\item This will execute app playbook only if the node is running \textbf{RedHat} family  
\item Update app.yml to restrict role to be run only on RedHat platform.
\end{itemize}

%= lang:text
\begin{code}
  ---
    - name: Playbook to configure App Servers
      hosts: app
      become: true
      vars:
        fav:
          fruit: mango
      roles:
      - { role: apache, when: ansible_os_family == 'RedHat' }
```  

  * Let's run this code  
  ```
  ansible-playbook site.yml
  ```
  [Output]  
  ```
  TASK [setup] *******************************************************************
ok: [192.168.61.12]
ok: [192.168.61.13]

TASK [base : create admin user] ************************************************
skipping: [192.168.61.12]
skipping: [192.168.61.13]

TASK [base : remove dojo] ******************************************************
skipping: [192.168.61.12]
skipping: [192.168.61.13]

TASK [base : install tree] *****************************************************
skipping: [192.168.61.12]
skipping: [192.168.61.13]

TASK [base : install ntp] ******************************************************
skipping: [192.168.61.12]
skipping: [192.168.61.13]

TASK [base : start ntp service] ************************************************
skipping: [192.168.61.12]
skipping: [192.168.61.13]

TASK [apache : Installing Apache...] *******************************************
skipping: [192.168.61.12]
skipping: [192.168.61.13]

TASK [apache : Starting Apache...] *********************************************
skipping: [192.168.61.12]
skipping: [192.168.61.13]

TASK [apache : Creating configuration from templates...] ***********************
skipping: [192.168.61.12]
skipping: [192.168.61.13]

TASK [apache : Copying index.html file...] *************************************
skipping: [192.168.61.12]
skipping: [192.168.61.13]

  ```  

**Exercise**: Try using **Debian** instead of **RedHat** . You shall see app role being skipped altogether. Don't forget to put it back after you try this out.


### Iterations  
#### Iteration over list  
* Create a list of packages  
  * Let us create the following list of packages in base role.  
  * Edit *roles/base/defaults/main.yml* and put  
  ```
---
# packages list
demolist:
  packages:
    - atk
    - flac
    - eggdbus
    - pixman
    - polkit

  ```  
  * Also edit *roles/base/tasks/main.yml* to iterate over this list of items and install packages

\end{code}
  - name: install a list of packages
    yum:
      name: ``\{\{ item \}\}''
    with\_items: \{\{ demolist.packages \}\}

\kode{ 
  * Let's check the output
 }
  TASK [base : install a list of packages] \textbf{**}\textbf{**}\textbf{**}\textbf{**}\textbf{**}\textbf{**}***
changed: [192.168.61.12] =\textgreater{} (item=[u'atk', u'flac', u'eggdbus', u'polkit', u'pixman'])
changed: [192.168.61.13] =\textgreater{} (item=[u'atk', u'flac', u'eggdbus', u'polkit', u'pixman'])

```\newline
\#\#\#\# Iterating over a Hash Table/Dictionary
  * This iteration can be done with using \textbf{with\_dict} statement, let us see how\newline
  * Edit \emph{group\_vars/all} file from the \textbf{parent directory} and define a dictionary of mysql databases and users to be created

%= lang:text
\begin{code}
---
  fav:
    color: blue
    fruit: peach
  mysql_bind: "{{ ansible_eth0.ipv4.address }}"
  mysql:
    databases:
      infinity:
        state: present
      peace:
        state: present
    users:
      dojo:
        pass: PassWord@1234
        host: '%'
        priv: '*.*:ALL'
        state: present
      koko:
        pass: f8Usg3ord@1we28
        host: '%'
        priv: '*.*:ALL'
        state: present

\end{code}
  * \textbf{Append} the following iteration in \emph{roles/mysql/tasks/config.yml}

%= lang:text
\begin{code}
  - name: create mysql databases
    mysql_db:
      name: "{{ item.key }}"
      state: "{{ item.value.state }}"
    with_dict: "{{ mysql['databases'] }}"

  - name: create mysql users
    mysql_user:
      name: "{{ item.key }}"
      host: "{{ item.value.host }}"
      password: "{{ item.value.pass }}"
      priv: "{{ item.value.priv }}"
      state: "{{ item.value.state }}"
    with_dict: "{{ mysql['users'] }}"

```  
  * Execute the *db* playbook to verify the output  
   ```
    ansible-playbook db.yml
   ```

## Exercises
  * Define dictionary of properties for a new database user  in group_vars/all. Observe if it gets created automatically  output by running db.yml playbook. Validate if the user is been actually present by logging on to the mysql server and checking status.

  * Update index.html.j2 to iterate over the dictionary of favorites and generate html content to display it instead of adding multiple lines.

  * Define a hash/dictionary  of apache virtual hosts to be created  and create a template which would iterate over that dictionary and create vhost configurations.

  * Learn about what else you could loop over, as well as how to do so by reading this document http://docs.ansible.com/ansible/playbooks_loops.html#id12
\end{code}